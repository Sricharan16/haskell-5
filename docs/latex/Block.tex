\haddockmoduleheading{Block}
\label{module:Block}
\haddockbeginheader
{\haddockverb\begin{verbatim}
module Block (
    Block,  validposition,  blockcolor,  randomblock,  rotateclock, 
    rotateanti,  shapeT,  shapeJ,  shapeS,  shapeZ,  shapeA,  shapeI,  shapeL, 
    shapeO,  blockcheck
  ) where\end{verbatim}}
\haddockendheader

\begin{quote}
{\haddockverb\begin{verbatim}
This creates a type Block and enumerates it to different possible shapes and randomly selects one of them to insert 
at the top of the game view\end{verbatim}}
\end{quote}

\begin{haddockdesc}
\item[\begin{tabular}{@{}l}
data\ Block
\end{tabular}]\haddockbegindoc
\haddockid{Block} is a data type of List of Integer pairs(coordinates) and a color of the block \par

\end{haddockdesc}
\begin{haddockdesc}
\item[\begin{tabular}{@{}l}
instance\ Show\ Block
\end{tabular}]
\end{haddockdesc}
\begin{haddockdesc}
\item[\begin{tabular}{@{}l}
validposition
\end{tabular}]\haddockbegindoc
\haddockbeginargs
\haddockdecltt{::} & \haddockdecltt{(Int, Int)} & The 'Int,Int' argument is the block position  \\
                                                  \haddockdecltt{->} & \haddockdecltt{Block} & the Block is input Block \\
                                                                                               \haddockdecltt{->} & \haddockdecltt{Bool} & the return value is true if it is valid else false \\
\end{tabulary}\par
the \haddockid{validposition} checks whether the position of the block is within the bounds of the grid or not\par

\end{haddockdesc}
\begin{haddockdesc}
\item[\begin{tabular}{@{}l}
blockcolor\ ::\ Block\ ->\ Color
\end{tabular}]\haddockbegindoc
the \haddockid{blockcolor} takes a block as input and assigns a color to it \par

\end{haddockdesc}
\begin{haddockdesc}
\item[\begin{tabular}{@{}l}
randomblock\ ::\ Double\ ->\ Block
\end{tabular}]\haddockbegindoc
the \haddockid{randomblock} takes a random seed and returns a block corresponding to it \par

\end{haddockdesc}
\begin{haddockdesc}
\item[\begin{tabular}{@{}l}
rotateclock\ ::\ Block\ ->\ Block
\end{tabular}]\haddockbegindoc
the \haddockid{rotateclock} takes a block and rotates it clockwise and returns the rotated block \par

\end{haddockdesc}
\begin{haddockdesc}
\item[\begin{tabular}{@{}l}
rotateanti\ ::\ Block\ ->\ Block
\end{tabular}]\haddockbegindoc
the \haddockid{rotateanti} takes a block and rotates it anti clockwise and returns the rotated block \par

\end{haddockdesc}
\begin{haddockdesc}
\item[\begin{tabular}{@{}l}
shapeT\ ::\ Block
\end{tabular}]\haddockbegindoc
\haddockid{shapeT} defines a shape of T \par

\end{haddockdesc}
\begin{haddockdesc}
\item[\begin{tabular}{@{}l}
shapeJ\ ::\ Block
\end{tabular}]\haddockbegindoc
\haddockid{shapeJ} defines a mirror view of the L shape\par

\end{haddockdesc}
\begin{haddockdesc}
\item[\begin{tabular}{@{}l}
shapeS\ ::\ Block
\end{tabular}]\haddockbegindoc
\haddockid{shapeS} defines a mirror view of the z shape\par

\end{haddockdesc}
\begin{haddockdesc}
\item[\begin{tabular}{@{}l}
shapeZ\ ::\ Block
\end{tabular}]\haddockbegindoc
\haddockid{shapeZ} defines a 90degrees rotated Z\par

\end{haddockdesc}
\begin{haddockdesc}
\item[\begin{tabular}{@{}l}
shapeA\ ::\ Block
\end{tabular}]\haddockbegindoc
\haddockid{shapeA} defines a single block of size 1x1\par

\end{haddockdesc}
\begin{haddockdesc}
\item[\begin{tabular}{@{}l}
shapeI\ ::\ Block
\end{tabular}]\haddockbegindoc
\haddockid{shapeI} defines a shape of 1x4 rectangle\par

\end{haddockdesc}
\begin{haddockdesc}
\item[\begin{tabular}{@{}l}
shapeL\ ::\ Block
\end{tabular}]\haddockbegindoc
\haddockid{shapeL} defines a shape of L\par

\end{haddockdesc}
\begin{haddockdesc}
\item[\begin{tabular}{@{}l}
shapeO\ ::\ Block
\end{tabular}]\haddockbegindoc
\haddockid{shapeO} defines a square of size 2x2\par

\end{haddockdesc}
\begin{haddockdesc}
\item[\begin{tabular}{@{}l}
blockcheck\ ::\ (Int,\ Int)\ ->\ Block\ ->\ Bool
\end{tabular}]\haddockbegindoc
the \haddockid{blockcheck} takes a coordinate point and a block and checks whether the point lies in the block or not \par

\end{haddockdesc}