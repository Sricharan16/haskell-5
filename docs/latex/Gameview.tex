\haddockmoduleheading{Gameview}
\label{module:Gameview}
\haddockbeginheader
{\haddockverb\begin{verbatim}
module Gameview (
    Grid,  emptygrid,  atompos,  blockrender,  doescollide,  rownumber, 
    atomindex,  Atom(Empty, FilledWith),  atomcol,  clearFilledRows
  ) where\end{verbatim}}
\haddockendheader

\begin{quote}
{\haddockverb\begin{verbatim}
This creates a Grid as a list of rows where each row is a list of atoms and also has functions to clear a filled row and check 
whether the blocks are colliding\end{verbatim}}
\end{quote}

\begin{haddockdesc}
\item[\begin{tabular}{@{}l}
data\ Grid
\end{tabular}]\haddockbegindoc
\haddockid{Grid} is a List of rows representing the playing space \par

\end{haddockdesc}
\begin{haddockdesc}
\item[\begin{tabular}{@{}l}
instance\ Show\ Grid
\end{tabular}]
\end{haddockdesc}
\begin{haddockdesc}
\item[\begin{tabular}{@{}l}
emptygrid\ ::\ Grid
\end{tabular}]\haddockbegindoc
\haddockid{emptygrid} is a list of empty rows \par

\end{haddockdesc}
\begin{haddockdesc}
\item[\begin{tabular}{@{}l}
atompos\ ::\ Grid\ ->\ {\char 91}(Int,\ Int,\ Atom){\char 93}
\end{tabular}]\haddockbegindoc
\haddockid{atompos} assigns the atom coordinates to all the atoms in the grid\par

\end{haddockdesc}
\begin{haddockdesc}
\item[\begin{tabular}{@{}l}
blockrender\ ::\ Block\ ->\ (Int,\ Int)\ ->\ Grid\ ->\ Grid
\end{tabular}]\haddockbegindoc
\haddockid{blockrender} takes a block and a coordinate point and a grid and returns a grid with the given block in it at the specified location \par

\end{haddockdesc}
\begin{haddockdesc}
\item[\begin{tabular}{@{}l}
doescollide\ ::\ Block\ ->\ (Int,\ Int)\ ->\ Grid\ ->\ Bool
\end{tabular}]\haddockbegindoc
\haddockid{doescollide} takes a block and a coordinate and a grid and returns true if the given block collides with any atom in the grid\par

\end{haddockdesc}
\begin{haddockdesc}
\item[\begin{tabular}{@{}l}
rownumber\ ::\ Grid\ ->\ {\char 91}(Int,\ Row){\char 93}
\end{tabular}]\haddockbegindoc
\haddockid{rownumber} assigns the row numbers to the rows \par

\end{haddockdesc}
\begin{haddockdesc}
\item[\begin{tabular}{@{}l}
atomindex\ ::\ Row\ ->\ {\char 91}(Int,\ Atom){\char 93}
\end{tabular}]\haddockbegindoc
\haddockid{atomindex} assigns the atom index to the atoms in a single row\par

\end{haddockdesc}
\begin{haddockdesc}
\item[\begin{tabular}{@{}l}
data\ Atom
\end{tabular}]\haddockbegindoc
\haddockbeginconstrs
\haddockdecltt{=} & \haddockdecltt{Empty} & \\
\haddockdecltt{|} & \haddockdecltt{FilledWith Color} & \\
\end{tabulary}\par
the \haddockid{Atom} is the basic unit of a block which is filled with a particular color \par

\end{haddockdesc}
\begin{haddockdesc}
\item[\begin{tabular}{@{}l}
instance\ Eq\ Atom\\instance\ Show\ Atom
\end{tabular}]
\end{haddockdesc}
\begin{haddockdesc}
\item[\begin{tabular}{@{}l}
atomcol\ ::\ Atom\ ->\ Color
\end{tabular}]\haddockbegindoc
the \haddockid{atomcol} takes an atom and assigns a color to it \par

\end{haddockdesc}
\begin{haddockdesc}
\item[\begin{tabular}{@{}l}
clearFilledRows\ ::\ Grid\ ->\ (Grid,\ Int)
\end{tabular}]\haddockbegindoc
\haddockid{clearFilledRows} takes the grid and returns a grid with the filled rows removed and the number of filled rows removed\par

\end{haddockdesc}